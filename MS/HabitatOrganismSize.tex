\documentclass[10pt]{article}
\usepackage{graphicx}
% \usepackage{fullpage}
\usepackage{setspace}
\usepackage{natbib}
%%%%% \usepackage{lineno}
\usepackage[margin=3.2cm]{geometry}
\usepackage[singlelinecheck=off]{caption}
\usepackage{lineno}
\setcounter{secnumdepth}{-1}


\begin{document}
\title{The strength of habitat filtering and organism size} 
\author{A. Andrew M. MacDonald, Diane
  S. Srivastava, Vinicius Farjalla}
%\date{November 2012}
\begin{spacing}{2}
\maketitle

\linenumbers

\section{Introduction}

Previous observational studies \cite{Farjalla2012} have shown that
larger bodied organisms show a greater degree of habitat filtering:
there is greater environmental structure as body size increases from
bacteria (weak), zooplankton (intermediate) and insects (strong).
This is especially interesting as these different ``environments'' are
caused by variation in habitat preference among several bromeliad
species -- some species prefer shrub cover, while others are found
centimeters away in full sun.  

In this experiment we will experimentally test these observations by
homogenizing these communities, removing variation both within and
among communities as much as possible.  We hypothesize that as habitat
filtering acts, via mortality and -- for non-metamorphosing organisms
-- reduced fecundity, we will observe a 'relaxation' of the community
towards the orginal species composition.  We hypothesize that this
relaxation will be greatest for insects, less for zooplankton and
nearly zero for bacteria.

\section{Methods}

\subsection{experimental design}
\label{sec:expdesig}

As discussed, I'm going to attempt three separate experiments,
representing a gradient in the extremity of habitat differences:
between-bromeliad species, between habitats (same species) and between
sizes (same species and habitat).
The following protocol describes the between-species part of the
experiment.  I'll use a similar technique for the remaining work.

Gather 30 bromeliads of approximately the same size, from three
species \emph{Aechmea nudicaulis}, \emph{Aechmea ligulata} and
\emph{Vriesea }.  \textbf{Nicolas believes getting all three to be the same
size is impossible, Vinicus and Juliana think it merely very
difficult.}

\paragraph{First step: obtain bromeliads.}

Wash all bromeliads, hang upsidedown to dry, and then place in
restinga, enclosed in a screen mesh big enough to keep out most
diptera (i.e. mosquito netting).  Bromeliads can be grouped in spatial
blocks, with each 'block' occuring on the same face of a different
shrub patch and containing one bromeliad from each species.  Species
will be placed in suitable habitats (e.g. exposed or inside the patch,
depending on bromeliad species). Bromeliads will be assigned to blocks
to minimize within-block variation in maximum volume.

\paragraph{second step: obtain insects, zooplankton and bacteria}

Insects, zooplankton and bacteria will be obtained separately from the
bromeliads in the experiment.  This is to allow for the longer time
needed to collect replicate bromeliads of the right sizes; it also
allows all the starting communities to be standardized to the same
species composition, rather than allow this to vary from block to
block.  Additionally, this allows the 'before' condition to be
quantified by a few representative subsamples of the whole mixture (in
the case of bacteria and zooplankton) rather than for each block
separately \emph{(Vinicus's suggestion.  what do you think?)}.
Starting communities of the three functional groups will be collected
in different ways:

\begin{itemize}
\item Macroinverts: First, calculate a 'representative'
community. Then, collect the necessary insects by pipetting/dissecting
bromeliads and distribute identical communities to every replicate
bromeliad.
\item zooplankton: Begin by collecting equal volumes of water from
  each of the three bromeliad species.  Then, use a fine sieve to
  concentrate this water, keeping sources separate. Visually estimate
  the densities of each of these three concentrations (using a scope
  that we plan to wheedle out of a NUPEM lab).  Combine them in
  concentrations that result in a community that is approx. per-capita
  equal parts of all three starting communities (this prevents the
  starting community from bearing a very strong resemblece to (a/any)
  bromeliad species which have high zoop densities)
\item bacteria: use a large pipette to mix up water in a leaf well and
  extract it.  Keep water from different species separate.  Measure
  out equal amounts of water from each of the three species, filter to
  remove zooplankton and mix.
\end{itemize}

Ideally, the water placed in each bromeliad at the start of the
experiment will consist as much as possible of zooplankton and
bacteria 'solution'.

Questions:

insects: how to calculate the starting community?  From Nicolas's
data?  If so, should only use the very few observational plants in the
same size range as the experimental plants?  Nicolas argues that,
since he found little effect of size in his analysis, the data could
be calculated from \emph{ALL} his observations from the relevant
species, ignoring size.  Andrew is unconvinced, and believes size to
be too important to species interactions and composition -- instead,
suggest we select some of the collected bromeliads to be censused
carefully, and these observations added to relevant (ie same-sized)
ones from Nicolas's data to calculate a plausible starting community.

zooplankton
The (over-elaborate?) defense against different zoop densities in
different bromeliad species may not be necessary; we are going to talk
to Viviane about it and also look at her data.





\subsection{Analysis}

\section{Results}

\end{spacing}

\bibliographystyle{../../../Writing/sysbio3} 
\bibliography{references.bib}
\end{document}